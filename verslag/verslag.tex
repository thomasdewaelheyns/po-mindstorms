\documentclass[12pt,a4paper]{report}

\usepackage{geometry}
\usepackage{graphicx}
\usepackage{longtable}
\usepackage{pgfgantt}
\usepackage[dutch]{babel}
\usepackage{url}
\usepackage{pgf,pgfplots}
\usetikzlibrary{fit,calc}
\usepgfplotslibrary{external}

\geometry{a4paper}

\newcommand{\boxplot}[6]{%
	%#1: center, #2: median, #3: 1/4 quartile, #4: 3/4 quartile, #5: min, #6: max
	\filldraw[fill=white,line width=0.2mm] let \n{boxxl}={#1-0.1}, \n{boxxr}={#1+0.1} in (axis cs:\n{boxxl},#3) rectangle (axis cs:\n{boxxr},#4);   % draw the box
	\draw[line width=0.2mm, color=red] let \n{boxxl}={#1-0.1}, \n{boxxr}={#1+0.1} in (axis cs:\n{boxxl},#2) -- (axis cs:\n{boxxr},#2);             	% median
	\draw[line width=0.2mm] (axis cs:#1,#4) -- (axis cs:#1,#6);                                                                           							% bar up
	\draw[line width=0.2mm] let \n{whiskerl}={#1-0.025}, \n{whiskerr}={#1+0.025} in (axis cs:\n{whiskerl},#6) -- (axis cs:\n{whiskerr},#6);        % upper quartile
	\draw[line width=0.2mm] (axis cs:#1,#3) -- (axis cs:#1,#5);                                                                           							% bar down
	\draw[line width=0.2mm] let \n{whiskerl}={#1-0.025}, \n{whiskerr}={#1+0.025} in (axis cs:\n{whiskerl},#5) -- (axis cs:\n{whiskerr},#5);        % lower quartile
}


\begin{document}

%  Titelblad

\begin{titlepage}

\fontsize{12pt}{14pt}\selectfont

\begin{center}

\vspace{1cm}

\fontsize{14pt}{17pt}\selectfont
% De Faculteit:
\textsc{P\&O Computerwetenschappen - Verslag \\Team Platinum}
\fontsize{12pt}{14pt}\selectfont
\vspace{0.3cm}

\vspace{1.2cm}

%Het academiejaar: aanpassen!
Academiejaar 2011--2012

\vspace{2.8cm}

\fontsize{17.28pt}{21pt}\selectfont

% De titel van de thesis:
{\textsc{Pac-Man in de echte wereld,\\ met behulp van Lego Mindstorms}}

\fontseries{m}
\fontsize{12pt}{14pt}\selectfont

\vspace{2cm}

\includegraphics[height=10cm]{resources/Logo-Kul}

\end{center}
\end{titlepage}

\thispagestyle{empty}

\tableofcontents

\begin{abstract}
Tijdens de uitwerking van dit jaar-overschrijdend groepswerk, bouwen wij een autonome robot met behulp van Lego Mindstorms. Voor de programmatie wordt beroep gedaan op Lejos, een Open Bron project dat een minimale JAVA virtuele machine heeft gemaakt die de plaats kan innemen van de standaard programmatie omgeving van Lego.

Naast de doelstelling om kennis te maken met het ontwikkelen van een autonome robot, willen we in dit project ook ervaring opdoen ivm het werken in team verband aan een middelgroot softwareproject. Hierbij zijn organisatie van werk, planning, analyse, architectuur,... belangrijke begrippen.

Hierbij komt in het tweede semester de belangrijke opgave van inter-teamcollaboratie, het traject van de autonome robot blijft immers behouden. Om deze collaboratie in goede banen te leiden werd een scheidsrechtercommissie in het leven geroepen om de beslissingen te nemen die voor alle teams van belang zijn. Het doel van dit semester ligt in het insluiten van een pacmanrobot bestuurd door het didactisch team, dit met behulp van vier autonome ghosts

In het eerste semester werkten we reeds in een parallel traject aan het ontwikkelen van een simulatoromgeving. Aan de hand van deze omgeving waren we in staat om met meerdere teamleden in parallel een robot te ontwikkelen zonder nood aan een fysieke. Nu is het ontwikkelen van deze simulator deel geworden van de verdere opdracht en dienen we dus de nodige aanpassingen te doen om het pacman-spel in de computer te simuleren. Het gebruik van een simulator is evident, aangezien we niet kunnen verwachten alle fysieke testen uit te voeren met de 4 teams tegelijk.
\end{abstract}

\chapter{Inleiding}

Het project wordt zoals in het eerste semester begeleidt door het gebruik van tussentijdse demo's. Het finale doel is om zo optimaal mogelijk samen te werken met vier teams om de Pac-Man in te sluiten, hoewel we ten alle tijden autonoom beslissingen nemen.

Het doel van de eerste demo is een versimpelde vorm van het einddoel waarbij we een stilstaande Pac-Man zoeken binnen het doolhof. De simulator dient ook reeds werkend te zijn bij deze vertoning. Voor de tweede demo mag de comissie beslissen welke doelstellingen passen in het verdere proces richting het einddoel.

Duidelijk is dat we er in geslaagd zijn om zeer herbruikbare code te produceren in het eerste semester. Iedereen is ondertussen vertrouwd geraakt met de volledige architectuur. Aangezien we duidelijk te maken hebben met een grote diversiteit aan kleine taken, gebeurt het grootste deel van de taakverdeling week op week.

Het verslag is duidelijk verschillend opgebouwd ten opzichte van het vorige semester en kiest voor een onderverdeling in subsecties per demo binnen een context van hoofdstukken. De opbouw van deze hoofdstukken is duidelijk erg gelijkend hoewel natuurlijk de inhoud verandert is tov het eerste semester. Nieuw zijn natuurlijk de stukken over de strategie ivm de opgegeven spelomgeving, over de collaboratie en over de finale analyse van het jaarproject.

\chapter{Probleemstelling}

De doelstellingen van de demo's worden hier achtereenvolgens beschreven, zij liggen in de lijn van het ontwerpproces richting een autonome ghost, die in collaboratie met de drie anderen probeert de Pac-Man in te sluiten.

\section{Demo 1}

Het probleem dat we eerst aanpakken is het zo efficient mogelijk in kaart brengen van een onbekende omgeving met behulp van de vier ghosts, belangrijk hierbij is dat de in beeld gebrachte wereld overeenkomt met de realiteit. Tijdens de demo gebruiken we een hybride simulator, onze eigen robot bestaat in de fysieke wereld, de andere drie zijn slechts virtueel.\\ De enige voorkennis die we hebben is onze eigen globale startpositie. Verder staat de Pac-Man stil in het parcours en dient hij enkel gevonden te worden. We dienen ook aan te tonen dat het door de commissie afgesproken communicatieprotocol geimplementeerd is.\\Elk team wordt tijdens de demo afzonderlijk beoordeelt.
 
\section{Demo 2}
 
Dit dient nog afgesproken te worden door de commissie.

\section{Demo 3}

\subsection{Verkenning}

De robots rijden in een op voorhand onbekend doolhof. Alle info die verzameld wordt door de vier ghosts draagt bij tot het creeren van een individuele map die mogelijk inconsistenties bevat. Alle juiste info over het doolhof zal direct bijdragen tot het hoofddoel, namelijk het vangen van de pac-man. Elke robot communiceert dus zijn gevonden info en probeert zo veel mogelijk informatie te verzamelen over de rond hem onbekende wereld, het falen van ��n van de robotten kan reeds fataal zijn voor het snel kunnen opbouwen van een verkende omgeving. We proberen dus een zo robuust mogelijke robot te maken die zo efficient mogelijk zijn eigen info verzameld.

\subsection{Pac-Man}

Het Pac-Man spel zelf kunnen we winnen door alle vluchtwegen van de Pac-Man af te sluiten. Dit houdt in dat alle omliggende sectoren door een ghost bezet wordt of afgeschermd worden door een muur. Men wint enkel als alle robots die meewerken aan de insluiting ook effectief aangeven dat ze gewonnen hebben via hun grafische interface. De probleemstelling vraagt dus om effectief samen te werken en zowel consensus te bereiken over de verzamelde informatie als over de te volgen strategie.


\chapter{Scheidsrechtercomissie}

Deze commissie staat in voor het nemen van beslissingen die betrekking hebben op:
\begin{itemize}
	\item De verdere regels waarmee de spelomgeving beperkt wordt.
	\item Afspraken die noodzakelijk zijn voor de samenwerking van de teams.
\end{itemize}

\section{Spelregels}

\subsection{SpelWereld}

De spelwereld bestaat uit aangepaste panelen uit het eerste semester. Er is een raster van witte lijnen geintroduceerd waar mogelijks muren kunnen staan. Dit komt neer op een mogelijke herschaling van de panelen met een factor vier. Deze nieuwe minimumeenheid aan oppervlakte noemen we sectoren. Deze zijn belangrijk voor de plaatsbepaling binnen het doolhof. De doolhof is volledig ommuurd en de absolute afmetingen worden op voorhand gegeven.

\subsection{Pac-Man}

De Pac-Man wordt zichbaar gemaakt door een Infraroodbeacon dat kan opgemerkt worden met behulp van de voorziene IR-sensor. Het didactisch team kiest waar deze vertrekt en bestuurt hem tijdens de demo's. Hij mag enkel bewegen op sectoren waar zich geen ghosts beevinden.\\De beweging van Pac-Man is beperkt tot het rechtdoor oversteken van de witte lijnen die de sectoren scheiden.

\subsection{Ghosts}

De ghosts vertrekken op de vier hoekpunten van de doolhof, zodat de verkenning van de doolhof zo snel mogelijk van start kan gaan. De orientatie van de robot is onbekend hoewel de hoek tov het assenstel wel een veelvoud is van 90 graden.


\section{CommunicatieProtocol}

Te vinden in pdf.
\subsection{Voordelen}

\subsection{Nadelen}


\chapter{Robot}

\section{Demo 1}

\subsection{Fysiek ontwerp}
Door de veranderde specificaties van het doolhof en de extra infrarood sensor, moesten enkele wijzigingen gemaakt worden aan de fysieke robot. Zo zijn de druksensoren verwijderd om een sensor poort vrij te maken voor de infrarood sensor. Verder werd de hoogte van de robot iets verlaagd door het verplaatsen van de motor die verantwoordelijk is voor het draaien van de sonar. Naast een grotere stabiliteit doordat het zwaartepunt van de robot lager bij de grond ligt staat ook de motor stabieler, waardoor nauwkeurigere metingen mogelijk zijn. Aangezien het doolhof minder breed is (sectoren zijn 40 centimeter in plaats van 80) zijn de extra wielen weggehaald om het geheel iets slanker te maken wat een eenvoudigere navigatie mogelijk maakt.
\subsection{Testen}
Ir Sensor
\section{Demo 2}
\section{Demo 3}

\chapter{SoftwareDesign}

\section{Demo 1}

\subsection{Robot}

\subsection{PC}

\section{Demo 2}
\section{Demo 3}

\chapter{Strategie}
\section{Strategie van Robot}

\subsection{Achtervolgstrategie}
Wanneer de Pacman gezien wordt stuurt deze een "scent" uit. Deze plant zich dan langs de gekende sectoren voort. Dit wordt gedaan met het gemiddelde van de 4 omliggende sectoren, indien er een muur staat is de waarde 0. Zo ontstaat er een hoogte-kaart met aan de top een Pacman die moet beklommen worden door de Ghosts. Deze hoogte-kaart wordt vaak herberekend en de geur van een Pacman zal dus snel verdwijnen. Als een Pacman meerdere keren gezien wordt zal deze ook een sterkere geur uitsturen en dus meer Ghosts aantrekken.

Iedere Ghost probeert de waarde van het vakje te maximaliseren en stapt zo steeds dichterbij. Om te voorkomen dat robots allemaal dezelfde weg volgen en enkel achter Pacman lopen slorpt iedere Ghost de geur op door de waarde van zijn huidig vakje op 0 te zetten. Hierdoor ontstaat er een dal rond iedere Ghost en zullen de anderen een omweg zoeken. Dit zorgt er ook voor dat de Ghosts elkaar van nature uit ontwijken en niet zullen botsen.


\subsection{Verkenstrategie}
Het verkenningsalgoritme is een variant op het achtervolgstrategie. Hier is er niet ??n bron die een geur uitstuurt, maar iedere onbekende sector stuurt een geur uit. De robot zal hierdoor altijd naar het dichtstbijzijnde en het grootste onverkende stuk gaan. Omdat de ene Ghost een andere eventueel kan insluiten in een verkend doodlopend stuk slorpen deze maar een stuk op. Hierdoor zal de robot nog steeds proberen te ontsnappen.

\chapter{Simulator}
\section{Demo 1}
\subsection{Virtualisatie}
\subsection{Granulariteit}
\subsection{Testen}
\section{Demo 2}
\section{Demo 3}

\chapter{Conclusies}
\section{Demo 1}
\section{Demo 2}
\section{Demo 3}
\chapter{Procesbeschrijving}

\chapter{Werkverdeling}

\begin{longtable}{l l}
\caption{Focus van elk team lid} \\ [0.5ex]
%This is the header for the first page of the table...
\hline\hline
Teamlid & Focus \\ [0.5ex]
\hline 
\endfirsthead
%This is the header for the remaining page(s) of the table...
\multicolumn{2}{c}{{\tablename} \thetable{} -- Vervolg} \\[0.5ex]
\hline \hline
Teamlid & Focus \\ [0.5ex]
\hline 
\endhead
Michiel 		& 	(Coordinator) Bluetooth communicatie, calibratie, finale navigator. \\
Florian 		&	RobotAPI, lijnvolger, opvolgen unit testen, finale navigator. \\
Ruben 		&	RobotAPI, veelhoek algoritme, muurvolger, sonar-navigator.\\
Thomas 		&	RobotAPI, lichtsensor, barcode-lezer, fat client. \\
Daniel 		&	Testen, barcode-lezer, fat client. \\
Christophe 	&	(Secretaris) Verslag, simulator, model \& navigator, web-client. \\
\hline
\label{tab:focus}
\end{longtable}

\begin{longtable}{l r r r r r r}
\caption{Tijdsbesteding per team lid per week} \\
%This is the header for the first page of the table...
\hline\hline
 & Michiel & Florian & Ruben & Thomas & Daniel & Christophe \\
\hline 
\endfirsthead
%This is the header for the remaining page(s) of the table...
\multicolumn{4}{c}{{\tablename} \thetable{} -- Vervolg} \\
\hline \hline
 & Michiel & Florian & Ruben & Thomas & Daniel & Christophe \\
\hline 
\endhead
totaal    & 121 & 90 & 91 & 76 & 78 & 128 \\
\hline
week 1 & 10 & 10 & 10 & 10 & 10 & 15 \\
week 2 & 9 & 7 & 8 & 7 & 9 & 16 \\
week 3 & 8 & 8 & 8 & 8 & 5 & 12 \\
week 4 & 9 & 7 & 5 & 8 & 5 & 17 \\
week 5 & 42 & 15 & 18 & 6 & 15 & 34 \\
week 6  & 15 & 11 & 12 & 18 & 15 & 5 \\
week 7 & 5 & 11 & 10 & 9 & 9 & 9 \\
week 8 & 10 & 9 & 12 & 5 & 5 & 9 \\
week 9 & 13 & 12 & 8 & 5 & 5 & 11\\
week 10 \\
week 11 \\
\hline
\label{tab:tijdsregistratie}
\end{longtable}

\chapter{KritischeAnalyse}

Dit deel wordt later toegevoegd.

\appendix

\chapter{Beoordelingen}

\chapter{Grafische User Interface}

\chapter{Klasse Diagrammen}


\chapter{Planning}
\label{appendix:planning}

\begin{figure}[htbp]
\centering
\begin{tikzpicture}
	\begin{ganttchart}[ 
	  y unit title=0.6cm,
	  y unit chart=0.5cm,
	  vgrid,
	  bar height=.5,
	  group right shift=0,
	  group top shift=.6,
	  group height=.3]{22}
\gantttitle{Oktober}{10} 	   	\gantttitle{November}{8} 	    \gantttitle{December}{4} \\
\gantttitlelist{3,10,17,24,31}{2}	\gantttitlelist{7,14,21,28}{2}  \gantttitlelist{5,12}{2} \\
\ganttgroup{1. Beweging}{1}{5} \\
\ganttbar{1.1. API Tijd}{2}{4} \\
\ganttbar{1.2. API Omw.}{2}{4} \\
\ganttbar{1.3. Tijd}{4}{5} \\
\ganttbar{1.4. Omw.}{4}{5} \\
\ganttbar{1.5. tests}{5}{5} \\
\ganttbar{2. Veelhoek}{3}{5} \\
\ganttbar{3. LCD}{2}{5} \\
\ganttmilestone{Demo 1}{5} \ganttnewline
\ganttgroup{4. Verslag}{1}{20} \\
\ganttbar{4.1. Demo 1}{2}{5} \\
\ganttbar{4.2. Demo 2}{8}{12} \\
\ganttbar{4.3. Finaal}{18}{20} \\
\ganttgroup{5. Comm.}{3}{20} \\
\ganttgroup{5.1. Robot}{3}{20} \\
\ganttbar{5.1.1. Log}{3}{11} \\
\ganttbar{5.1.2. Commando}{11}{20} \\
\ganttgroup{5.2. PC}{5}{20} \\
\ganttbar{5.2.1. Log}{5}{11} \\
\ganttbar{5.2.2. Commando}{11}{20} \\
\ganttbar{6. Model}{3}{18} \\
\end{ganttchart}
\end{tikzpicture}
\caption{Planning}
\label{fig:planning}
\end{figure}

\begin{figure}[htbp]
\centering
\begin{tikzpicture}
	\begin{ganttchart}[ y unit title=0.6cm, y unit chart=0.5cm,  vgrid,
	  bar height=.5,
	  group right shift=0,
	  group top shift=.6,
	  group height=.3]{22}
\gantttitle{Oktober}{10} 	   	\gantttitle{November}{8} 	    \gantttitle{December}{4} \\
\gantttitlelist{3,10,17,24,31}{2}	\gantttitlelist{7,14,21,28}{2}  \gantttitlelist{5,12}{2} \\
\ganttgroup{7. Navigator}{5}{18} \\
\ganttbar{7.1. Abstract}{5}{11} \\
\ganttbar{7.2. Design}{11}{18} \\
\ganttbar{7.3. Impl.}{13}{18} \\
\ganttbar{7.4. Tests}{15}{18} \\
\ganttbar{8. Simulator}{3}{13} \\
\ganttbar{9. Log server}{5}{11} \\
\ganttbar{10. Syslog}{7}{10} \\
\ganttbar{11. DB server}{7}{10} \\
\ganttgroup{12. Fat client}{9}{16} \\
\ganttbar{12.1. Status}{9}{14} \\
\ganttbar{12.2. Commando}{10}{16} \\
\ganttgroup{13. SOA}{9}{12} \\
\ganttbar{13.1. REST}{9}{11} \\
\ganttbar{13.2. web client}{10}{12} \\
\ganttbar{14. Smart client}{12}{18} \\
\ganttbar{15. Calibratie}{12}{18} \\
\ganttbar{16. Unit tests}{2}{18} \\
\ganttgroup{17. Lijnvolger}{5}{10} \\
\ganttbar{17.1. Onderzoek}{5}{8} \\
\ganttbar{17.2. Sensor}{7}{9} \\
\ganttbar{17.3. Impl.}{9}{10} \\
\ganttgroup{18. Muurvolger}{5}{10} \\
\ganttbar{18.1. Onderzoek}{5}{8} \\
\ganttbar{18.2. Sensor}{7}{9} \\
\ganttbar{18.3. Impl.}{9}{10} \\
\ganttgroup{19. Barcode}{5}{10} \\
\ganttbar{19.1. Onderzoek}{5}{8} \\
\ganttbar{19.2. Detectie}{7}{9} \\
\ganttbar{19.3. Impl.}{9}{10} \\
\ganttmilestone{Demo 2}{10} \ganttnewline
\ganttmilestone{Finale Demo}{18} \ganttnewline
\ganttbar{20. Finaal verslag}{19}{20} \\
\end{ganttchart}
\end{tikzpicture}
\caption{Planning (vervolg)}
\label{fig:planning2}
\end{figure}

\end{document}
